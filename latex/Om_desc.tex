\documentclass[12pt,a4paper]{article}
\usepackage[left=2cm,right=2cm,top=2cm,bottom=2cm]{geometry}
\usepackage[utf8]{inputenc}
\usepackage[english]{babel}
\usepackage{amsmath}
\usepackage{amsfonts}
\usepackage{amssymb}
\usepackage{hyperref}
\usepackage{color}
\author{Felix Plasser}
\title{1TDM Descriptors}

\newcommand{\om}[1]{\omega_{\textrm{#1}}}
\newcommand{\doi}[1]{\href{http://dx.doi.org/#1}{\textcolor{blue}{\bf DOI: #1}}}

\begin{document}
\section{Standard 1TDM descriptors}
Standard fragment based descriptors of the one-particle transition density matrix (1TDM).
$A$ and $B$ are the indices of the fragments as defined in \texttt{at\_lists} .
For more information see Ref~\cite{DMAT}.
\\~\\
\begin{tabular}{clc}
\hline 
\textbf{Key} & \textbf{Formula} & \textbf{Description} \\ 
\hline
\\
Om & $\Omega = \sum_{AB}\Omega_{AB}$ & norm of the 1TDM, single-exc. character \\*[1.5ex]
POSi & $\om{POSi}=\Omega^{-1}\sum_A A\left(\sum_B \Omega_{AB}\right)$ & initial/hole position	 \\*[1.5ex]
POSf & $\om{POSf}=\Omega^{-1}\sum_B B\left(\sum_A \Omega_{AB}\right)$ & final/electron position	 \\*[1.5ex]
POS & $\om{POS}=(\om{POSi} + \om{POSf})/2$ & average position \\*[1.2ex]
CT  & $\om{CT}=\Omega^{-1}\sum_{A,B\neq A}\Omega_{AB}$ & charge transfer \\*[1.5ex]
CT2  & $\om{CT2}=\Omega^{-1}\sum_{A,B\not\in \lbrace A-1,A,A+1\rbrace}\Omega_{AB}$ & CT to second nearest neighbour \\*[1.5ex]
CTnt & $\om{CTnt}=\om{POSf} - \om{POSi}$ & net CT distance \\*[1.2ex]
PRi & $\om{PRi}=\Omega^{2}/\sum_A\left(\sum_B \Omega_{AB}\right)^2$ & initial/hole delocalization \\*[1.5ex]
PRf & $\om{PRf}=\Omega^{2}/\sum_B\left(\sum_A \Omega_{AB}\right)^2$ & final/electron delocalization \\*[1.5ex]
PR & $\om{PR}=(\om{PRi}+\om{PRf})/2$ & arithmetic mean delocalization\\*[1.2ex]
PRh & $\om{PRh}=2/(\om{PRi}^{-1}+\om{PRf}^{-1})$ & harmonic mean delocalization\\*[1.2ex]
COH & $\om{COH}=\Omega^{2}/\left(\om{PR}\sum_{AB}\Omega_{AB}{}^2\right)$ & coherence length\\*[1.5ex]
COHh & $\om{COHh}=\Omega^{2}/\left(\om{PRh}\sum_{AB}\Omega_{AB}{}^2\right)$ & harmonic mean coherence length\\*[1.5ex]
\hline 
\end{tabular} 

\section{Transition metal complexes}
Specific 1TDM descriptors to be used for transition metal complexes \cite{Ircomp}.
$M$ is the central metal atom (specified as the first fragment in \texttt{at\_lists}), $L,L'$ are the ligands.
\\~\\
\begin{tabular}{clc}
\hline 
\textbf{Key} & \textbf{Formula} & \textbf{Description} \\ 
\hline
\\
MC & $\om{MC}=\Omega^{-1}\Omega_{MM}$ & metal centered contribution \\*[1.5ex]
LC & $\om{LC}=\Omega^{-1}\sum_L\Omega_{LL}$ & locally excited ligand centered contribution \\*[1.5ex]
MLCT & $\om{MLCT}=\Omega^{-1}\sum_L\Omega_{ML}$ & metal-to-ligand CT contribution \\*[1.5ex]
LMCT & $\om{LMCT}=\Omega^{-1}\sum_L\Omega_{LM}$ & ligand-to-metal CT contribution\\*[1.5ex]
LLCT & $\om{LLCT}=\Omega^{-1}\sum_{L\neq L'}\Omega_{LL'}$ & ligand-to-ligand CT contribution\\*[1.5ex]
\hline 
\end{tabular}

\begin{thebibliography}{9}
\bibitem{DMAT} F. Plasser and H. Lischka \textit{JCTC} \doi{10.1021/ct300307c}.
\bibitem{Ircomp} F. Plasser and A. Dreuw \textit{JPCA} \doi{10.1021/jp5122917}.
\end{thebibliography}

\end{document}